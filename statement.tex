\documentclass[11pt]{article}
\usepackage{mathtools,hyperref,booktabs,fullpage, txfonts}
\usepackage[amssymb,cdot]{SIunits}
\usepackage[utopia]{mathdesign}     

\usepackage[table]{xcolor}
\usepackage{amsmath}
\usepackage{hyperref}
\usepackage{longtable}
\usepackage{fullpage}
 
\definecolor{lightgray}{gray}{0.93}

\pagestyle{empty}
\setlength\parindent{0pt}
\renewcommand{\thefootnote}{\fnsymbol{footnote}}
 
\makeatletter
\renewcommand\section{\@startsection{section}{1}{\z@}%
                                  {-3.5ex \@plus -1ex \@minus -.2ex}%
                                  {2.3ex \@plus.2ex}%
                                  {\normalfont\bfseries}}
\makeatother


\begin{document}

{\large
  \begin{center}
    {\bf ME 701 -- Development of Computer Applications In Mechanical Engineering \\ 
         Homework 1 - Statment}
  \end{center}
}
 

\section*{Problem 1 -- Open-Source Software}

Think of the things you do routinely on a computer that require 
specific software packages.  Find an 
open-source solution from the software repository  
for one of these activities and tell me about it in 100 words or less.
For example, I used to do lots of audio recording when I was in 
high school (not {\it that} long ago) and used special (and 
pretty expensive) tools like 
Cakewalk Sonar.  Since then, I've found an 
open-source package for doing multitrack 
recording called Ardour that doesn't have all the bells and 
whistles but, because I can program in C++ and the 
source code is available, I could, in theory,
create any such whistles I need.  {\bf Note}: You may not
describe anything already discussed in class (e.g., the LibreOffice suite
or Octave).


\section*{Problem 2 -- Command-Line Utilities}

\begin{enumerate}
\item Figure out how to display information about your CPU via the 
      command line.  This should include at least the processor 
      speed and the number of cores.  Describe your command(s) in your
      writeup, and include the output as a separate, well-named text file.
      (Hint: redirection is helpful.)
\item Figure out how to list the programs that use the most 
      amount of (1) processing and (2) memory.  Describe your command(s) 
      in your writeup, and include the output as separate,
      well-named text files. 
\end{enumerate}

 

\end{document}
